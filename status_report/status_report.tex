    
\documentclass[11pt]{article}
\usepackage{times}
\usepackage{enumitem}
\setitemize{noitemsep,topsep=0pt,parsep=0pt,partopsep=0pt}
    \usepackage{fullpage}
    
    \title{Self Driving Car Using the Browser on a Smartphone}
    \author{Kirstin Ritchie 2389589r}

    \begin{document}
    \maketitle
    
    
     

\section{Status report}

\subsection{Proposal}\label{proposal}

\subsubsection{Motivation}\label{motivation}

Using a smartphone camera that is being accessed through the browser on a webpage 
allows for large accessibility to all consumers wanting to make or develop a 
self-driving robot car. In comparison to a similar product, the Amazon Deep Racer 
(https://aws.amazon.com/deepracer/) allows customers to learn various machine 
learning techniques but at the expensive cost of purchasing the robot itself 
and other additives such as software, frameworks, and tailored tutorials. The 
self-driving car robot that I plan to implement will demonstrate if it is 
possible to make a fully self-driving car robot using current technology.

\subsubsection{Aims}\label{aims}

This project will work towards developing a fully self-driving robot car 
that makes use of current technology. This project is broken up into stages 
which increases in difficulty.

\begin{enumerate}
\item Follow an object 
\item Follow a face 
\item Follow a black line 
\item Fully self-driving 
\end{enumerate}

Each stage will be implemented using Javascript along with various other frameworks 
and APIs to aid development. The ultimate goal of this project is to check if it is 
possible to create a fully self-driving car using a personal smartphone that 
is supported by current technology. 

\subsection{Progress}\label{progress}

\begin{itemize}
\item Main language chosen for this project: Javascript, HTML, CSS. 
\item Research conducted for the first three stages (follow an object, 
face, and black line).
\item Implementation of basic control panel to move the robot.
\item Implementation of object following using the p5.js library.
\item Implementation of face tracking using ml5js and MediaPipe face-API.
\item Implementation of the feature to switch camera environments in 
object tracking. 
\item Recorded demos which demonstrate the first two stages working. 
\item Created a solid plan for the third stage (follow a black line).
\end{itemize}


\subsection{Problems and risks}\label{problems-and-risks}

\subsubsection{Problems}\label{problems}

Some of the issues that were encountered during the project so far:

\begin{itemize}
\item Compatability with iOS is limited. Using WebBLE as an initial 
solution proved to be difficult when beginning implementation of stages 
where camera access is required. WebBLE does not provide the capability to 
access the camera on an iOS device. 
\item Initial hosting platform (p5js editor) was suddenly unable to support 
the web Bluetooth connection using UART.js.
\item Implementation of following face stage using ml5js produces a slow video 
feed which causes issues with following the face: very laggy, slow updates to 
the robot.
\item Switching the camera in the following face stage using MediaPipe does 
not currently work: plan to possibly resolve the issue before the beginning of 
semester 2. 
\end{itemize}


\subsubsection{Risks}\label{risks}

\begin{itemize}
\item Still to create a plan for the third stage using machine learning techniques:
plan to consider some options before the beginning of the second semester 
\item Unsure of how to evaluate the success of this project: Should success be 
demonstrated through recorded demos? Should there be more comprehensive testing to 
measure the success?: planning to research ways to measure success to provide a 
good evaluation for this project.
\end{itemize}

\subsection{Plan}\label{plan}

\subsubsection{Semester 2}

NOTE: Writing for the dissertation will be continuous throughout semester 2.

\begin{itemize}
    \item 
        Week 1-2: Update supervisor on progress made during the winter break. Continue with the 
        development of the third stage and start considering some machine learning techniques 
        that could be used in the third stage. \textbf{Deliverable:} Finalise implementation of 
        the third stage along with demos to display progress.
    \item 
        Week 3-4: Finalize the code for the third stage and begin research on the fourth stage. 
        Devise a detailed plan for the fourth stage. \textbf{Deliverable:} PDF of all conducted research 
        and an outline of the detailed plan for the fourth stage. 
    \item 
        Week 5-7: Continue and finish the implementation of the fourth stage (if possible) and start 
        possibly conduct some evaluations to test the success of the project. \textbf{Deliverable:} 
        Finished implementation of the fourth stage (again if possible) and an outline of some 
        evaluation tests I would consider conducting. 
    \item 
        Week 8-9: Begin writing up parts of the dissertation but also continue with any needed implementation 
        or testing. \textbf{Deliverable:} Recorded demos of all stages working along with any required testing 
        to measure the performance of each stage. Any written parts of the dissertation is sent to the supervisor 
        for feedback.
    \item 
        Week 10: 'Tidy up' of all code and update the GitHub repository by adding all visualization and 
        produce a first draft of the dissertation. \textbf{Deliverable:} First draft submitted 
        to supervisor for feedback or approval two weeks (or more) before the deadline. 
\end{itemize}

\end{document}
